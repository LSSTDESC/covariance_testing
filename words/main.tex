\RequirePackage{docswitch}
% \flag is set by the user, through the makefile:
%    make note
%    make apj
% etc.
\setjournal{\flag}

\documentclass[twocolumn]{\docclass}

%%%% Scott's macros
\newcommand{\sfig}[2]{
	\includegraphics[width=#2]{#1}
}
\newcommand{\Sfig}[2]{
	\begin{figure}[thbp]
		\sfig{../Figures/#1.pdf}{\columnwidth}
		\caption{{\small #2}}
		\label{fig:#1}
	\end{figure}
}
\newcommand{\Swide}[2]{
	\begin{figure*}[thbp]
		\sfig{../Figures/#1.pdf}{.8\textwidth}
		\caption{{\small #2}}
		\label{fig:#1}
	\end{figure*}
}
\newcommand{\Sswide}[2]{
	\begin{figure*}[thbp]
		\sfig{../Figures/#1.pdf}{.7\textwidth}
		\caption{{\small #2}}
		\label{fig:#1}
	\end{figure*}
}
\newcommand{\Svwide}[2]{
	\begin{figure*}[thbp]
		\sfig{../Figures/#1.pdf}{\textwidth}
		\caption{{\small #2}}
		\label{fig:#1}
	\end{figure*}
}

\newcommand{\Spng}[2]{
	\begin{figure}[thbp]
		\sfig{../Figures/#1.png}{0.95\columnwidth}
		\caption{{\small #2}}
		\label{fig:#1}
	\end{figure}
}
\newcommand{\Rf}[1]{\ref{fig:#1}}
\newcommand{\rf}[1]{\ref{fig:#1}}
\newcommand{\ec}[1]{Eq.~(\ref{eq:#1})}
\newcommand{\ecalt}[1]{Eq.~\ref{eq:#1}}
\newcommand{\Ec}[1]{(\ref{eq:#1})}
\newcommand{\eeec}[3]{Eqs.~(\ref{eq:#1}, \ref{eq:#2}, \ref{eq:#3})}
\newcommand{\eql}[1]{\label{eq:#1}}
\newcommand\be{\begin{equation}}
	\newcommand\ee{\end{equation}}
\def\bea{\begin{eqnarray}}
	\def\eea{\end{eqnarray}}
% \def\bea{\begin{eqnarray}}
% \def\eea{\end{eqnarray}}
\def\svs{\nonumber\\}

% You could also define the document class directly
%\documentclass[]{emulateapj}

% Custom commands from LSST DESC, see texmf/styles/lsstdesc_macros.sty
\usepackage{lsstdesc_macros}
\newcommand\scott[1]{{\bf [Scott: #1]}}
\usepackage{graphicx}
\graphicspath{{./}{./figures/}}
\bibliographystyle{apj}

% Add your own macros here:



% ======================================================================

\begin{document}
	
	\title{Covariance Testing}
	
	\maketitlepre
	
	\begin{abstract}
		
		There  are a number of codes that compute covariance matrices analytically; the plan is to use these to build TJPCov. In this project, we start along the path of comparing these different codes, building up a suite of tools that can be used to compare covariance matrices. We expect these tools to be useful not only for converging on a single accurate code for computing covariance matrices but also more generally for understanding which parts of the covariance matrix carry the most information (and therefore need the most attention to get right) and which are not relevant (so for example matrices that are not positive definite may still be usable if the negative eigenmodes are not relevant).
	\end{abstract}
	
	% Keywords are ignored in the LSST DESC Note style:
	\dockeys{}
	
	\maketitlepost
	
	% ----------------------------------------------------------------------
	% 
	
	\section{Introduction}
	\label{sec:intro}
	
Almost all analyses of cosmological data use a likelihood function that requires a covariance matrix. If the data set has $N$ data points, the covariance matrix is a symmetric $N\times N$ and captures the errors in the measurements, including those that are correlated from one point to another. As data sets grow, the number of elements in the covariance matrix is becoming too large to verify each individual element. For example, we will focus on cosmic shear measurements from the Dark Energy Survey (DES)~\cite{Troxel:2017xyo}; in that case, the data vector has 227 elements, varying with angular separation, and different pairs of tomographic redshift bins. In this case, then, the number of independent elements of the covariance matrix is $227\times 228/2=25,878$. Which of these are most important to get right? What should the tolerance be when comparing different techniques or even different codes using the same technique? How many simulations need to be run in order to obtain accurate enough covariance matrices? 

Here attempt to address these questions by considering several increasingly complex methods of identifying the elements of the covariance matrix that are most relevant. Although we will consider only the one example of DES cosmic shear measurements and projections, we think that our conclusions will be useful more generally. In \S\ref{sec:methods}, we describe the data set used and the pair of covariance matrices constructed in order to test different validation schemes. The next 3 sections walk through increasingly complex ways of comparing these two matrices. Before diving in, we comment that a very simple way to compare two covariance matrices is to see whether they obtain the same final parameter constraints. We will use this metric a bit throughout but want to emphasize here that our goals are a bit more ambitious: simply finding out that two covariance matrices give different results is a black-box approach. One cannot identify the source of the disagreement. The methods described here aim to go a bit deeper to try to understand where the key differences arise and which differences are most important.
		
	\section{DES Cosmic Shear: Data and Analysis}
	\label{sec:methods}
	
	In this section, we explain the methodology used for each of the tests carried out for the analysis and comparison of the covariance matrices. Our analysis is carried out using cosmic shear statistics $\xi_\pm(\theta)$, focusing on the Year 1 results of the Dark Energy Survey \citep{Troxel:2017xyo,Abbott:2018cms} (DESY1) and also on predictions for DES Year 3 (DESY3). The data is divided into four tomographic redshift bins spanning the interval $0.20 < z < 1.30$, which yields 10 bin-pair combinations, each one containing 20 angular bins between 2.5 and 250 arcmin. We thus begin with 200 data points for each $\xi_+(\theta)$ and $\xi_-(\theta)$, giving 400 in total. We then apply the angular cuts described in \citep{Abbott:2018cms}, which removes the scales most sensitive to baryonic effects; this leaves 167 points for $\xi_+(\theta)$ and 60 for $\xi_-(\theta)$, resulting in 227 data points, which corresponds to the aforementioned $227 \times 227$ covariance matrix. 
		
Table~\ref{tab:constraints} shows the 16 parameters varied and the priors placed on them. To perform cosmological parameter inference we use the {\tt CosmoSIS} \citep{Zuntz:2015med} pipeline, while employing the {\tt MultiNest} \citep{Feroz:2009fhb} sampler to explore the parameter space, with 500 {\tt livepoints}, {\tt efficiency} set to 0.3, {\tt tolerance} to 0.1 and {\tt constant efficiency} set to False.

	
	
	\begin{table}
		\centering
		\begin{tabular} { l c} 
			\hline
			\hline
			Parameter							& Prior	\\ \hline
			Cosmological & \\ [1ex]
			$\Omega_m$						& $\mathcal{U}(0.1, 0.9)$		\\
			$A_s \times 10^9$					& $\mathcal{U}(0.5, 5)$		\\
			$H_0 \mathrm{(km s^{-1} Mpc^{-1})}$	& $\mathcal{U}(55, 91)$		\\
			$\Omega_b$						& $\mathcal{U}(0.03, 0.07)$	\\
			$\Omega_\nu h^2$					& $\mathcal{U}(0.0005, 0.01)$	\\
			$n_s$							& $\mathcal{U}(0.87, 1.07)$	\\ [1ex]
			\hline
			Astrophysical & \\ [1ex]
			$A$								& $\mathcal{U}(-5, 5)$ \\
			$\eta$							& $\mathcal{U}(-5, 5)$ \\ [1ex]
			\hline
			Systematic & \\ [1ex]
			$m^i$							& $\mathcal{G}(0.012, 0.023)$	 \\
			$\Delta z^1$						& $\mathcal{G}(-0.001, 0.016)$	 \\
			$\Delta z^2$						& $\mathcal{G}(-0.019, 0.013)$	 \\
			$\Delta z^3$						& $\mathcal{G}(0.009, 0.011)$	 \\
			$\Delta z^4$						& $\mathcal{G}(-0.018, 0.022)$	 \\ [1ex]
			\hline
			\hline
		\end{tabular}
		\caption{List of the priors used in the analysis for parameter constraints ($\mathcal{U}$ denotes flat in the given range and $\mathcal{G}$ is gaussian with mean equal to its first argument and dispersion equal to its second). For the cosmological parameters, we fix $w = -1.0$, $\Omega_k =  0.0$ and $\tau =  0.08$. The astrophysical parameters are associated with the intrinsic alignment, they follow the relation $A(z) = A[(1+z)/1.62]^{\eta}$. Lastly, for systematics we have $m^i$ corresponding to the shear calibration and  $\Delta z^i$ for the source photo-$z$ shift, with $i = 1, 4$ in both cases.}
		\label{tab:constraints}
	\end{table}
	
	The covariance matrices are obtained using two different codes: 
	\begin{itemize}
	\item {\tt Cosmolike} \citep{Krause:2016jvl} (CL) was used in the initial DESY1 analysis and here we simply take the Y1 covariance and divide by the ratio of the Y3/Y1 areas (roughly a factor of 3) to generate a Y3 covariance matrix
	\item We ran the code used to analyse the KiDS-450 survey \citep{Kohlinger:2017sxk} (BJ) using DESY3 parameters and tomographic bins (all taken to be the same as Y1 apart from the area), but using only the gaussian part. 
	\end{itemize}
	Thus, throughout, the covariance labels CL and BJ differ for several reasons: first, they are two independent codes and, second, BJ was run with very simple settings. To be clear, the BJ code does contain all the functionality in CL, but, in order to accentuate the differences, we ran with the simplest settings. The ensuing larger differences will help us assess different validation techniques.
	
\figref{y3-comparison} shows the results for the projected cosmological constraints for DESY3. These projections use the same data vector and cuts as was used for Y1, but the two different covariance matrices. The best-fit values agree within 1-$\sigma$ but the constraints are up to $25\%$ broader when the CL covariance matrix is used. \scott{not sure this is clear: can we simply quote the 1-sigma errors?}
This is also true for the other parameters, where, on average, constraints obtained with CL are about $18\%$ wider.	
		
	\begin{figure}
		\includegraphics[width=0.9\columnwidth]{Y3-comparison.pdf}
		\caption{Constraints on cosmological parameters $\Omega_m$ and $\sigma_8$ for two covariance matrices produced for cosmic shear for DESY3. The purple curve is for CL while the blue is for BJ. The constraints are about $18\%$ larger for the former, indicating that the two matrices have quantifiable differences between them. \label{fig:y3-comparison}}
	\end{figure}
	

	
\section{Element-by-element comparison}

	If there was only a single data point, then the covariance matrix would contain only a single number, and comparing two covariance matrices to try to understand why they get different constraints would be as simple as comparing these two numbers. 
The simplest generalization of this is to do an element-by-element comparison of the two covariance matrices.
We make a scatter plot of the elements of the two matrices in \figref{one-to-one}, where we can see that the elements of CL are, in general, larger than BJ's, differing by up to 3 orders of magnitude. In some ways, this is useful and reassuring, as it aligns with what we see in the parameter constraints: larger elements in the covariance matrix translates to less constraining power.
	
	\begin{figure}
		\includegraphics[width=0.9\columnwidth]{One-to-one-Y3_BJ-CL.pdf}
		\caption{Scatter plot of the elements of the covariance matrices BJ and CL. \scott{Explain color coding; also, see comment in text.}
			\label{fig:one-to-one}}
	\end{figure}
	
\scott{Can we identify the diagonal elements? Can we show where in the 227x227 space the ones with the biggest differences live?}
	
\section{Eigenvalues}
	
	The next simplest thing to try is to explore the eigenvalues of the covariance matrix. Each is associated with a linear combination of the data vector, or a \emph{mode}, and it is possible that identifying the modes that have the most discrepant eigenvalues will give guidance on how to reconcile differences.  We plot the eigenvalues for these matrices in \figref{coveigen}. At a first glance, both curves show reasonable agreement, with values differing only by an average of $\approx 15\%$, but we should keep in mind that the order of magnitude of the values vary greatly, which makes it difficult to compare them efficiently with this methodology, furthermore, it is not clear which elements are most important.
	

	\begin{figure}
		\includegraphics[width=0.9\columnwidth]{Eigenvalues/Eigenvalues_Y3_BJ-CL.pdf}
		\caption{A log plot showing the $227$ eigenvalues of CL (blue) and BJ (orange). \label{fig:coveigen}}
	\end{figure}


	
	The lowest eigenvalues would correspond to the least variance and therefore hold the most information, whereas the highest eigenvalues would implicate larger error and thus contribute significantly less, which implies that the most important elements of the matrix are those with the lowest eigenvalues. To verify this assumption, we need to look at the parameter constraints: removing the elements with the highest eigenvalues should not alter our results; the opposite, however, should give us broader constraints. Our procedure consists of first diagonalizing the covariance matrix in order to calculate its eigenvalues and then replacing the eigenvalues of interest with numbers of about nine orders of magnitude higher, thus removing their effective contribution; we then obtain a new covariance matrix with the modified eigenvalues. Finally, we perform a cosmological analysis with the new covariance matrix, to constrain the parameters of our model. 
	
		
	We perform two different analysis: the first consist of modifying the fifty lowest eigenvalues, and for the second, we take the fifty highest.  \figref{eigenvalues} shows the results of both approaches compared to constraints for the original covariance matrix. We can see in the Posterior Density Functions (pdfs) that the mean values of both parameters are well within the $2\sigma$ interval, which is also true for all the other parameters. As for the constraining power, the sizes of the constraints are within less than 5\% of each other for almost all of the parameters. Since the results are similar for both approaches, it becomes clear that ranking eigenvalues is not the most efficient way of evaluating where the most contributing elements of the covariance matrix lie.
	
	
	\begin{figure}
		\includegraphics[width=0.9\columnwidth]{Eigenvalues/Eigenvalues_wm-sigma8.pdf}
		\caption{Constraints on cosmological parameters $\Omega_m$ and $\sigma_8$ for the original DESY1 covariance matrix (in purple) and for two new covariance matrices obtained by setting the fifty highest eigenvalues of the original matrix to nine orders of magnitude higher (in blue), and by replacing the fifty lowest eigenvalues to nine order of magnitude lower (in magenta). \label{fig:eigenvalues}}
	\end{figure}
	
	% ----------------------------------------------------------------------
	
	\section{Signal-to-noise ratio}
	
	Another way of analysing covariance matrices is by obtaining the signal-to-noise ratio (SNR). A simple way of achieving this is to note that the total SNR squared is
	%Since the values of the data points vary by many orders of magnitude, the range of eigenvalues in \figref{evector} is a bit misleading. In addition, it is not clear which eigenvectors should be dropped: should it be those with large eigenvalues (noise) or small ones (the noise is noisy)? Ideally, what we would like is to create a list of eigenvectors ordered in some way and drop the least important. This ties into the section below on Shrinkage. A very simple way to do this is to note that the total signal to noise squared is
	\be
	\left(\frac{S}{N}\right)^2 = \sum_{ij} D_i C^{-1}_{ij} D_j\
	,\ee
	where $D_i$ and $D_j$ are the data points and $C$ the covariance matrix. If $C$ were diagonal, then the eigenvectors would simply be the data points themselves, and we could estimate the SNR squared expected in each mode by simply computing $T_i^2/C_{ii}$ where $T_i$ is the theoretical prediction. Then we could throw out the modes with the lowest SNR. Since $C$ is not diagonal, we have to first diagonalize it and then order the values. So, we write the expected SNR squared as
	%to do things a bit differently. Let's first define re-scaled data points
	\bea
	\left(\frac{S}{N}\right)^2 &=& \sum_{ij} T_i  C^{-1}_{ij} T_j\nonumber\\
	&=& \sum_{i} \frac{v_i^2}{\lambda_i}
	,\eea
	where $\lambda_i$ are the eigenvalues of the covariance matrix, which is diagonalized with the unitary matrix $U$, and the eigenvectors are 
	\be
	v_i\equiv U_{ij}^T T_j\
	.\ee
	This makes it very clear which modes should be kept and which should be dropped. Modes $v_i$ for which $v_i^2/\lambda_i$ is very small can be discarded. 
	
	After obtaining the SNR for the covariance matrix, we proceed to set the 50 lowest values to seven orders of magnitude lower, which is equivalent to increasing the noise (or decreasing the signal) of these modes. We then obtain a new covariance matrix with the corresponding modified SNR values. 
	
	
	The parameter constraints for this method are shown in \figref{signalnoise}, where we note that the mean values are different about 10\% and 4\% for $\Omega_m$ and $\sigma_8$, respectively, with 17\% broader constraints for the former, when compared to the results with the original covariance. This tells us that the modes removed are essential for obtaining better constraints on these parameters. 
	
	Another aspect that we must consider when applying this procedure is that the highest modes do not necessarily hold the most information for all of the parameters; this is evident when we analyse the SNR for each parameter individually. To illustrate this, we take
	\bea
	\left(\frac{\partial S/\partial p_\alpha}{N}\right)^2 = \sum_{i} \frac{(\partial v_i / \partial p_\alpha)^2}{\lambda_i}
	,\eea
	with
	\be
	v_i^\alpha \equiv U_{ij}^T \frac{\partial T_j}{\partial p_\alpha}
	,\ee
	where $\partial /\partial p_\alpha$ is the derivative with respect to each parameter $\theta$. This produces the SNR for each parameter of interest. The importance of this procedure is illustrated in \figref{signalnoise_cuts}, where we can see that while the highest SNR for all the parameters does indeed correspond to those that hold most information for parameters $\Omega_m$ and $A_s$, this is not the case, for the intrinsic alignment parameters $A$ and $\eta$. As a result, we may end up losing constraining power over these parameters when we modify the lowest values of SNR. While this is apparent in the aforementioned figure, it is not very clear when looking at the resulting constraints on these parameters because the constraining power of the data over these parameters is not very strong (errors of about $100\%$). As such, these results do not encourage us to use this method for identifying the most important elements of the covariance matrix.
	
	\begin{figure}
		\includegraphics[width=0.9\columnwidth]{SNR/SNR_wm-sigma8.pdf}
		\caption{Constraints on cosmological parameters $\Omega_m$ and $A_s$ for the original DESY1 covariance matrix (in purple) and for a covariance matrix (in blue) obtained by setting fifty elements corresponding to the lowest SNR to a value seven orders of magnitude lower, in order to evaluate their contribution to parameter constraints. \label{fig:signalnoise}}
	\end{figure}
	
	\begin{figure}
		\includegraphics[width=1\columnwidth]{SNR/SNR_cuts.pdf}
		\caption{Scatter plot for the relation between the signal to noise obtained with the covariance matrix for DESY1 for each parameter (x-axis) against that for the full set of parameters (y-axis). The derivatives are shown with respect to cosmological parameters $\Omega_m$ (blue) and $A_s$ (orange), and for the intrinsic alignment parameters $A$ (green) and $\eta$ (red). The purple rectangle spreads until the fiftieth lowest value of SNR, which corresponds to the values that are modified for parameter constraints. \label{fig:signalnoise_cuts}}
	\end{figure}
	
	% ----------------------------------------------------------------------

	% ----------------------------------------------------------------------
	
	\section{Shrinkage}
	\label{subsec:shrinkage}
	
	%There have been several methods proposed in the literature to compress the data vectors, extracting as much information as possible. Here we consider two: first compression at the map level \citep{Alonso:2017hhj}, where linear combination of the tomographic maps is used. If there are 4 tomographic bins, an uncompressed analysis would require ten separate 2-point functions (or 20 for cosmic shear), whereas a compression scheme leads to just a few uncorrelated maps. If there were 3 such maps, then only three 2-point functions would need to be used for the likelihood analysis.
	
	Our final test for comparing covariance matrices and extracting the most important elements, consists of shrinking the data vector and the covariance matrix. There have been several methods proposed in the literature to compress the data vectors, extracting as much information as possible \citep{Tegmark:1997maa, Joachimi:2017mnr,Gualdi:2018gmj}. Here we consider two: compression at the map level \citep{Alonso:2017hhj}, where linear combinations of the tomographic maps are used to gain maximum signal-to-noise ratio, before transferring this compression scheme into real-space to compress the 2-point function. The second method directly operates in the real-space \citep{Zablocki:2015zcm}, where the modes used are those that maximize the Fisher information for each of the cosmological parameters.
	
	Shrinking the data vector and its covariance matrix is equivalent to projecting the data vector onto a subspace of the original vector space. For example, if our data have a length of $n$ and we want to compress it into a lower dimension $m\leq n$, we need to come up with a basis transformation matrix $U_{m\times n}$ that has orthonormal row vectors, or mathematically speaking, $<U_{xi} \cdot U_{yi}>=\delta_{xy}$. Once we have a basis transformation matrix $U_{m\times n}$, we can compress the data vector $y$ and covariance matrix $C$ into filter data $\Tilde{y}$ and $\Tilde{C}$
	\bea
	\Tilde{y} &=& U y\\\nonumber
	\Tilde{C} &=&<\Tilde{y}  \Tilde{y}^T>\\
	&=&U<y y^T> U^T= U C U^T\
	.\eea
	For a data vector with length $m$, there are $m(m+1)/2$ elements in the covariance matrix to be evaluated. Once the data vector is shrunk to length $n$, elements in covariance matrices will shrink quadratically. Therefore, shrinking the covariance matrices is an effective way of comparing them.
	
	Here, we will demonstrate these two methods to compress the covariance matrices. In both cases, the data vector is shrunk by 90\%, which correspond to the covariance matrices being reduced by 99\%. 
	
	\subsection{Tomographic Compression}
	
	This compression method is based on Karhunen-Lo\'eve decomposition for the shear power spectrum suggested by \citep{Alonso:2017hhj} and later applied to real space 2-point function in \citep{Bellini:2019ssw} for CFHTLens survey. This method generally finds the eigenmode with most of the signal-to-noise ratio contribution to the power spectrum, and then transforms the 2-point function in real space based on this eigenmode.
	
	With  {\tt CosmoSIS}, we can generate the shear power spectrum $C_{\ell}$ of convergence $a_{\ell m}$ for a fiducial cosmology. The cosmology we choose is the best-fit value of the DES Year 1 results for cosmic shear only. With the shear power spectrum $C_{\ell}=S_{\ell}+N_{\ell}$ and its shape noise $N_{\ell}$, we can calculate the Karhunen-Lo\'eve (KL) modes matrix $E^p_{\ell}$ via a general eigenvalue problem:
	\be
	C_{\ell} E^p_{\ell} = \lambda ^p N_{\ell} E^p_{\ell}\
	,\ee
	and the new observable $b_{\ell m} = E_p \cdot N^{-1} a_{\ell m}$. We should note that $C_{\ell}$ is the power spectrum of the convergence of the weak lensing, and $E_p$ is the transformation of basis for the convergence. We should point out that these eigenmodes are uncorrelated, so the power spectrum of the new observable $D_{\ell}$ is a diagonal matrix, with 1+SNR of the corresponding eigenmodes on the diagonal elements,
	\be
	D_{\ell} =\ <b_{\ell m} b_{\ell m}^T>\ = E^p_{\ell} N^{-1} C_{\ell} N^{-1} E^{pT}_{\ell}\
	,\ee
	or, if we denote $E^p_{\ell} N^{-1}$ as $R_{\ell}$, we can write the compression in one simple linear combination of the $C_{\ell}$,
	\be
	D_{\ell} = R_{\ell}^i C_{\ell}^{ij} R_{\ell}^j = U_{\ell}^{ij} C_{\ell}^{ij}\
	.\ee
	The double summation weight $U_{\ell}^{ij}$ is what we use to perform tomographic compression. Since the KL-decomposed modes of shear power spectrum are uncorrelated, we can make a compression here by only taking the first one or two modes with the highest SNR. By doing so, we compress 10 tomographic combinations to 1 or 2.
	
	We want, however, to eventually compress the 2-point function data vector of DESY1. One possible way is to calculate the 2-point function of the KL mode of the shear power spectrum. We can calculate the 2-point function from the shear power spectrum by
	\be
	\xi_{\pm}^{ij}(\theta) = \int \frac{\ell d \ell }{2\pi}J_{0/4}(\ell \theta) C^{ij}(\ell)\
	.\ee
	In order to compress the 2-point function based on the compression of the $C_{\ell}$, we need to make sure that the scheme for $C_{\ell}$ is $\ell$-independent, that is to say, the 2-point correlation function of $D_{\ell}$, $\Tilde{\xi}_{\pm}(\theta)$, can be directly calculated from other 2-point functions. We then have,
	\bea
	\nonumber\Tilde{\xi}_{\pm}(\theta) &=& \int \frac{\ell d \ell }{2\pi}J_{0/4}(\ell \theta) D(\ell)\\\nonumber
	&=&\int \frac{\ell d \ell }{2\pi}J_{0/4}(\ell \theta) U^{ij}C^{ij}(\ell)\\
	&=&U^{ij}\xi_{\pm}^{ij}(\theta)\
	,\eea
	where $U^{ij}$, the $\ell$-independent compression weight is calculated by 
	\be
	U^{ij} = \frac{\int_{\ell _{\mathrm{min}}}^{\ell _{\mathrm{max}}} (2 \ell +1) U^{ij}_{\ell}}{\int_{\ell _{\mathrm{min}}}^{\ell _{\mathrm{max}}} (2 \ell +1)}\
	.\ee
	We make a more conservative angular cut than the angular cut discussed in \textcolor{red}{[ref]}, making sure that the cut for both $\xi_{\pm}$ are uniform in regard to tomographic combinations. For $\xi_+$, we consider an angular scale from $7.195^{\circ}$ to $250.0^{\circ}$. For $\xi_-$, the angular scale is from $90.579^{\circ}$ to $250.0^{\circ}$. Therefore, for the purpose of demonstrating KL-transform, the raw data vector has a length of 190, and by shrinking 10 tomographic combinations for each angle into 1 KL-mode, the data is shrunk to 19, and so the number of elements in the covariance matrices are reduced by 99\%.
	
	%We characterize a given element in the data vector by its angular and tomographic indices: $a_i = a_{\ell m,\alpha}$ where $l,m$ denote the indices corresponding to given spherical harmonics and $\alpha$ is a given tomographic bin, or equivalently $a_i = a_\alpha(\vec\theta)$ in real space. The compression occurs in terms of tomographic bins, so that 
	%\be
	%b_\mu(\vec\theta) = \sum_{\alpha} F_{\mu\alpha} a_\alpha(\theta)
	%\ee
	%where the $F$'s are chosen \footnote{Note that this definition of $F$ differs from that in \citep{Alonso:2017hhj} in that it includes the inverse of the noise matrix.} so that the correlation functions of the $b$'s are diagonal in tomographic space:
	%\bea
	%w_{\mu\nu}(\theta) &=& \langle b_\mu(\vec\theta_1) b_\nu(\vec\theta_2)\rangle\vert_{\vert\vec\theta_1-\vec\theta_2\vert\in\theta}
	%\svs
	% &=& \delta_{\mu,\nu} w_{\mu}(\theta).
	%\eea
	%Usually with $N_t$ tomographic bins, one must consider $N_t(N_t+1)/2$ correlation functions, but -- due to the transformation that diagonalizes the elements -- there are only $N_t$ correlation functions to consider (for a given $\theta$). Even better, these can be ordered by the information they contain, so fewer than $N_t$can be used. In the example used in \citep{Alonso:2017hhj}, 16 tomographic bins were assumed, so that the standard treatment would require 136 correlation functions, but only 3 were needed in order to extract accurate constraints. 
	
	%This method suggests a way of extracting the most important pieces of the full covariance matrix $C$. We simply compute the covariance matrix of the $w_{\mu}$ in terms of $C$ and keep only the most important terms. That is,
	%\bea
	%C^b_{\mu\nu} &\equiv& \langle (w_{\mu} -\bar w_\mu)\,(w_{\nu} -\bar w_\nu)\rangle
	%\svs
	% &=& \sum_{\alpha\alpha'\beta\beta'} F_{\mu\alpha}F_{\mu\alpha'}\, F_{\nu\beta}F_{\nu\beta'}\, C_{\alpha\alpha'\beta\beta'} .
	%\eea
	%Here the angular indices have been suppressed (there are two of them, one for each $w_\mu$), but the full tomographic complexity has been retained. The covariance matrix on the right includes a total of $N_t^2 \times N_t^2$ terms (some of which are equal because of symmetry) corresponding to all possible pairs of 2-point functions. But $C^b$ on the right contains only $N_t^2$ elements, and again these are ordered, so we can use only a subset of them. In the simplest case, where only one linear combination is needed so $\mu=\nu=1$, a single number (for each pair of angular bins) captures all the relevant information from the full covariance matrix.
	
		With  {\tt CosmoSIS}, we calculate the shear power spectrum $C_{\ell}$ of DES Year 1 with a fiducial cosmology at the best-fit parameters, and $\ell$-range $2-5000$. The left plot in \figref{ClDl}, shows the diagonal elements of the signal part and the noise part of $C_{\ell}$, while the right one shows the KL-transformed eigenmode $D_{\ell}$ of $C_{\ell}$. We can see that the first KL mode contains most of the SNR contribution to the power spectrum. If we want to recover most information, we can also include the second mode.
	
	\begin{figure*}
		\includegraphics[width=0.7\columnwidth]{Cl_pst.png}
		\qquad \qquad \qquad
		\includegraphics[width=0.7\columnwidth]{Dl_pst.png}
		\caption{\textbf{Left:} Shear power spectrum of DESY1. Solid lines are diagonal elements of the signal matrix $S_{\ell}$, and dashed lines are the diagonal elements of noise matrix $N_{\ell}$. \textbf{Right:} Signal to noise ratio matrix $D_\ell$ of KL-modes of the power spectrum on the left. \textcolor{red}{Make the plots colour blind friendly.} \label{fig:ClDl}}
	\end{figure*}
	
	In \figref{kl-mode}, we plot the normalized KL-eigenmode $E_\ell^i$ of $C_{\ell}$ and its corresponding $W^{ij}_\ell=E_\ell^i E_\ell^j$. Modes with different $\ell$ are plotted with different depth of the color. We can see that the KL-modes do not depend a lot on scale factor $\ell$ by a large portion, so we take the weighted average of the eigenmodes $E_\ell^p$ and its quadratic form $W_\ell$ over $\ell$'s and plot them with black lines. 
	
	\begin{figure*}
		\includegraphics[width=0.80\columnwidth]{epi.png}
		\includegraphics[width=1.02\columnwidth]{Wij.png}
		\caption{\textbf{Left:} Normalized KL-eigenmodes $E_\ell^p$ of the shear power spectrum $C_{\ell}$, darkness of the color is representing different $\ell$. \textbf{Right:} Transformation on tomographic bin combination $W_{ij}$ constructed by the KL-eigenmodes. Black lines are the weighted average of each mode. \label{fig:kl-mode}}
	\end{figure*}
	
	For different $\ell$, the KL-modes do vary by a slight amount. We also observe that tomographic bins with higher redshift gains more weight than those with low redshift. This is also shown by the weight on tomographic combination that the combination of bin 3 and bin 4 gains most of the weight to maximize signal to noise ratio. This agrees with the fact that the diagonal elements $C_\ell$ for low redshift is much less than those with high redshift. \textcolor{red}{Is my $N_\ell$ actually right?}
	
	We use only the first KL-mode to compress the tomographic combination for each angle $\theta$, i.e., 190 data points for 19 angles are compressed into 19 numbers.  This compression scheme is also used to compress the CL and BJ covariance matrices, and we plot them in \figref{comp-cov}.
	
	\begin{figure*}
		\includegraphics[width=0.68\columnwidth]{kl_comp_cl.png}
		\includegraphics[width=0.68\columnwidth]{kl_comp_bj.png}
		\includegraphics[width=0.54\columnwidth]{kl_scatter.png}
		\caption{\textbf{Left:} CL covariance matrix compressed by the first KL-mode. \textbf{Middle:} BJ covariance matrix compressed by the first KL-mode. \textbf{Right:} One-to-One scatter of the two compressed matrices \label{fig:comp-cov}}
	\end{figure*}
	
	If we now again make a one-to-one comparison of these two compressed covariance matrices. We can easily notice that the elevated clump at the bottom right for scatter of original matrices, which represent elements with great difference, are gone in the compressed covariance. Instead, the two covariance matrices just have a relative constant difference because of the different model they use. This shows that the divergence between CL and BJ covariance does not affect the overall signal to noise ratio a lot.

	
	\subsection{Linear combinations of the data vector}
	
	The compression here  takes place at the 2-point level \citep{Zablocki:2015zcm}, with the compressed data vector containing linear combinations of the many 2-point functions. In principle, this might work with only $N_p$ 2-point functions where $N_p$ is the number of free parameters, and each mode, or linear combination, contains all the information necessary about the parameter of interest. 
	
	For each parameter $p_\alpha$ that is varied, one captures a single linear mode
	\be
	y_\alpha = U_{\alpha i} D_i\
	,\ee
	where $D_i$ are the data points and the coefficients are defined as
	\be \label{eq:compression_scheme}
	U_{\alpha i} \equiv \frac{\partial T_j}{\partial p_\alpha} \, C^{-1}{}_{ji}\
	,\ee
	with $T_j$ being the theoretical prediction for the data point $D_j$.
	The now much smaller data set $\{y_\alpha\}$, which contains as few as $N_p$ data points, carries with it its own covariance matrix, with which the $\chi^2$ can be computed for each point in parameter space. Propagating through shows that this covariance matrix is related to the original $C_{ij}$ via
	\be
	C_{\alpha\beta} = U_{\alpha i} C_{ij} U^T_{j\beta}.
	\ee
	In our case, our covariance matrix is  $227 \times 227$, while the number of parameters needed to specify the model is only 16, so $C_{\alpha\beta}$ is a $16\times 16$ matrix. We have apparently captured from the initial set of $(227 \times 228)/2 = 25,878$ independent elements of the covariance matrix a small subset (only 136 in this case) of linear combinations of these 25k elements that really matter. If two covariance matrices give the same set of $C_{\alpha\beta}$, it should not matter whether any of the other eighty thousand elements differ from one another.
	
	
	\section{Results}
	\label{sec:results}
	
	Ultimately, what matters is how well the likelihood does at extracting parameter constraints. Since most analyses assume a Gaussian likelihood, this boils down to how well the contours in parameter space agree when computing the $\chi^2$ using two different covariance matrices.
	
	For brevity, we show only constraints on $\Omega_m$ and $\sigma_8$.
	
	
	%\figref{evector} shows an example of one of the eigenvectors, the one associated with the smallest eigenvalue. This low-eigenvalue mode picks up the differences between the correlation function at different angular scales (each vertical line delineates between 2-point functions of shears in different tomographic bin pairs).
	
	%	\begin{figure}
	%		\includegraphics[width=0.9\columnwidth]{evector.png}
	%		\caption{A simple scatter plot of elements of covariance matrices produced by two separate halo model codes. \label{fig:evector}}
	%	\end{figure}
	
	% ----------------------------------------------------------------------
	
		
	
	
	
	
	\figref{compressiony1} compares the constraints obtained for the compressed covariance and dataset with the full one. The mean values agree at the $2 \sigma$ level, with the exception of $\eta$, which is not very well constrained in either analysis. The results for the compressed covariance are about 0.5\% broader, which shows that the information loss is negligible. 
	
	\begin{figure}
		\includegraphics[width=0.9\columnwidth]{Compression/CompressionY1_wm-sigma8.pdf}
		\caption{Constraints on cosmological parameters $\Omega_m$ and $\sigma_8$ for the original DESY1 covariance matrix (in purple) and for the compressed one (in blue). \label{fig:compressiony1}}
	\end{figure}
	
	One relevant point in this analysis is at which point to take the derivative of each parameter. When we wish to compare the results of our compression scheme with those obtained with the full covariance matrix and data set, it is important to derivate each parameter at their respective mean value (obtained by performing the analysis with the full covariance matrix). The shape and variance of the posterior is not dependent on the derivative, but the best-fit value shifts according to the point where the derivative is taken.
	
	We also apply this methodology to comparing the covariance matrices of interest, i. e. CL and BJ. In order to do this, we take two different approaches: first, we assume that $U_{\alpha, i}$ is the same for both covariance matrices and we calculate it with BJ. The second approach is that each compression scheme must use the original covariance matrix that will be compressed, so that $U_{\alpha, i}$ will be different for each covariance matrix. We find that the mean values of the parameter contraints for the two methods agree to $1 \sigma$, which shows that they are equivalent to each other. \figref{correlation} is obtained for the first method, which will be the one adopted from here on, it shows the correlation matrix for BJ and CL, and that of the difference between them; we find this figure important because we can clearly see the difference between the two matrices by simply looking at only $(16 \times 17)/2$ elements, as opposed to having to analyse the larger correlation matrix for the full covariance matrices. It is also crucial that the matrices used for comparison here are those obtained via the same compression scheme, so that we can be sure that their differences are indeed only related to the differences in the original matrices.
	
	
	\begin{figure}
		\includegraphics[width=0.9\columnwidth]{Correlation_compression.pdf}
		\caption{The upper right and lower left plots display the correlation matrix for BJ and CL respectively, while the lower right is the difference between the two. \label{fig:correlation}}
	\end{figure}
	
	
	
	
	
	% ----------------------------------------------------------------------
	% ----------------------------------------------------------------------
	
	\section{Discussion}
	\label{sec:discussion}
	
	
	\textcolor{red}{By making the transformation described above, we will be able to reorganize the covariance matrices into cosmological informative blocks and uninformative blocks. Intrinsically, we can tolerate more errors in the uninformative blocks and have a stronger requirement on the informative blocks. This will put a lot of simulation time into more valuable work.}
	
	\textcolor{red}{The gold of the invertibility of the transformation is that we can first assign higher tolerance to C3 and assign lower tolerance to C1 and C2, then use the transformation matrix to recover the covariance matrices in the original basis. Then, we can compare the recovered matrix with the original one and check out how much tolerance is allowed on each element. Without invertibility, we cannot accomplish this task, which allows us to precisely quantify the tolerance on each element in the covariance matrices.}
	
	
	% ----------------------------------------------------------------------
	% ----------------------------------------------------------------------
	
	\section{Conclusion}
	\label{sec:conclusion}
	
	
	% ----------------------------------------------------------------------
	% ----------------------------------------------------------------------
	
	\subsection*{Acknowledgments}
	
	%%% Here is where you should add your specific acknowledgments, remembering that some standard thanks will be added via the \code{desc-tex/ack/*.tex} and \code{contributions.tex} files.
	
	%This paper has undergone internal review in the LSST Dark Energy Science Collaboration. % REQUIRED if true
	
		%T.F. and T.Z. contributed equally on writing the main paper as well as implementing the covariance comparison and compression. N.C. contributed to the compression code. All authors participated in the discussion and gave valuable suggestions.
	
	The contributions are listed below. T.F. contributed to the manuscript, led the analysis for the eigenvalues, SNR and MOPED, as well as the comparison of the compressed covariance matrices. T.Z. contributed to the manuscript, participated and contributed substantially to all analysis, and led the analysis for the tomographic compression. N.C. contributed to the compression code. S.D. proposed the project, the analyses, led the discussions and also contributed to writing and editing the manuscript. % Standard papers only: author contribution statements. For examples, see http://blogs.nature.com/nautilus/2007/11/post_12.html
	
	% This work used TBD kindly provided by Not-A-DESC Member and benefitted from comments by Another Non-DESC person.
	
	% Standard papers only: A.B.C. acknowledges support from grant 1234 from ...
	
	\input{desc-tex/ack/standard} % also available: key standard_short
	
	\textcolor{red}{Need to thank CAPES as well.}
	
	\appendix
		\section{Invertible Transformation and Tolerance Testing}
	
	In the last section, we shrink the data vector and covariance matrices and find the most cosmological-informative modes in both. However, in order to make a tolerance testing of each element in the covariance, we not only need the cosmological-informative modes but also the uninformative modes. Suppose we find the informative set of modes for cosmology, the modes that are orthogonal to them, or the complementary set of the informative one, form a set of modes that are cosmological-uninformative. 
	
	To find these modes, we start off with the compression scheme presented in Eq. \ref{eq:compression_scheme}, which will serve as the basis for a rotation matrix $W_\alpha$. We then use the Gram-Schmidt decomposition to create $227 - N_p$ vectors orthonormal to $U_{\alpha}$, thus obtaining a unitary $227 \times 227$ matrix. We then have,
	\be
	C'_{\alpha\beta} = W_{\alpha i} C_{ij} W^T_{j\beta}.
	\ee
	%When this transformation is applied to the original matrix, $C_{ij}$, the first $N_p \times N_p$ elements of the new matrix will be the same as we had in $C_{\alpha\beta}$. Since our transformation spreads out the information contained in the compressed matrix, the relevant values for parameter estimation will be contained in the first $N_p$ rows and columns of the new covariance matrix. This means that we will have $N_p \left( 2 \times 227 - N_p \right)$ important elements, which is about an order of magnitude less than in the case of the full covariance matrix.
	
	\begin{figure}[]
		\includegraphics[width=0.45\columnwidth]{Transformation_data.pdf}
		
		\bigskip
		
		\includegraphics[width=0.99\columnwidth]{Transformation_CM.pdf}
		\caption{Illustration of the invertible transformation, $W$. It's components, $U$ and $U'$ are the compression scheme described in Sections \ref{subsec:shrinkage} and orthogonal components obtained using the Gram-Schmidt decomposition, respectively. \textbf{Top:} Transformation of the data vector, where D1 corresponds to compressed set, and D2 are uninformative for parameter constraints. \textbf{Bottom:} Transformation of the covariance matrix. The last square on the right, $C'$ is divided into four blocks: C1 is the most informative to cosmology, C2 are also relevant to constraints, but on a lower scale, and, finally, C3 is irrelevant to the $\chi^2$ calculation. \label{fig:transformation}}
	\end{figure}
	
	With the invertible transformation $W$, shown in \figref{transformation}, the data vector and covariance matrix are transformed into sectional blocks. We will describe the meaning of each block here. The data vector is split into two blocks: D1 has the transformed data points that are sensitive to changes in the cosmological parameters, that is, the cosmological-sensitive data; D2 is generated by the modes that are orthogonal to D1, they will, in principle, be unaffected by changes to the cosmological parameters, which makes them uninformative for parameter constraints.
	
	The transformed covariance matrix is split into 4 blocks, with the off-diagonal ones being the transpose of each other. Block C1 is the variance and covariance of the cosmological-sensitive data, D1, it describes how well the cosmological-sensitive data is measured, and is, therefore, the one that contributes most to the $\chi^2$ calculation, and, consequently, to parameter constraints. Block C2 is the cross-correlation between D1 and D2, it is important for parameter constraints because it describe how D1 is affected by the uncertainty of D2.  Finally, C3 relates to the uninformative data, D2; it plays a minor role to $\chi^2$, so it affects the parameter constraints least.
	
	%With this invertible transformation W shown in \figref{illu-trans}, the data vector and covariance matrix are transformed into sectional blocks. We will describe the meaning of each blocks here. The data vector is split into two blocks. D1 is the block of transformed data points that are sensitive to the changes in cosmological parameter set, that is, the cosmological-sensitive data. D2 is generated by the modes that are orthogonal to the cosmological sensitive mode, they will in principle stays relatively constant when cosmological parameter changes and are uninformative for parameter constraints.
	
	%The transformed covariance matrix is split into 4 blocks, with the off-diagonal blocks being the transpose of each other, there are 3 different blocks. Block C1 is the variance and covariance of the cosmological informative data, or D1. This block contributes almost everything to the $\chi^2$ calculation. C1 describes how well the cosmological-sensitive data is measured, so the value in this block will affect the parameter constraints. Block C2 is the variance and covariance of the uninformative data, or D2. It plays a minor role in the $\chi^2$, so it will affect the parameter constraints less. C3 is the cross-correlation between D1 and D2, and it will affect the parameter constraints because it describes how D1 is affected by the uncertainty of D2.
	
	The next step is then to test the tolerance of different parts of the transformed covariance matrix. We first compare the results of increasing the error of each block separately by a factor of 100 and compare the results to constraints with the unmodified covariance, then we start by introducing smaller errors to the relevant blocks, C1 and C2 and analyse the corresponding increase in the parameter constraints.
		\section{Tolerance Testing}
	
	Given that the Block C1 contains the most relevant elements, it was expected that any changes to it would also modify the parameter constraints. It is clear in 	\figref{tolerance1000} that this was not the case. Multiplying the elements of Blocs C1 and C3 did not alter the constraints, whereas changes made to Blocks C2 and C1+C2 did. To explain this, we need to look at what happens to $\chi^2$ when using the transformed dataset and covariance matrix. We have,
		\be
	v_i^\alpha \equiv U_{ij}^T \frac{\partial T_j}{\partial p_\alpha}
	,\ee
	
	
	
	\begin{figure}
		%\includegraphics[width=0.9\columnwidth]{Multiply_error.pdf}
		\caption{The upper right and lower left plots display the correlation matrix for BJ and CL respectively, while the lower right is the difference between the two. \label{fig:tolerance1000}}
	\end{figure}
	

	
%		In this section, we will introduce the methodology of tolerance testing on the covariance matrices in the new basis. We will show that after the basis transformation, we need to compare tremendously less elements from two covariance matrices.
	
%	Generally speaking, the parameter constraints will get wider when part of the covariance matrix gets larger, because this means that the data is worse measured. In the normal basis, each data points varies with the parameter set, so the change of each element in the covariance matrix will affect the constraints. However, since we have found a basis where some of the data points change with parameter set while others do not, the elements that matters to the cosmological constraints will also shrink. To be specific, the C1 and C2 blocks in Figure~\rf{illu-trans} are the only parts that impact the parameter constraints. 
	
%	In order to illustrate this statement, we will first perform a sanity check. We will make each of the blocks in the covariance matrices a large number and run a chain with that covariance. We run 3 tests by making
%	1) C3,  
%	2) C2, 
%	3) C1 and C2,
%	100 times of itself/themselves. In principle, we should see that the constraints for test 1 will not change, but will get wider for test 2 and will totally blow up for test 3.
	
%	Furthermore, we want to know how minor uncertainty on the informative elements of the covariance matrix will affect the cosmological constraints.
	
	
	% ----------------------------------------------------------------------
	% ----------------------------------------------------------------------
	

	% This work used some telescope which is operated/funded by some agency or consortium or foundation ...
	
	% We acknowledge the use of An-External-Tool-like-NED-or-ADS.
	
	%{\it Facilities:} \facility{LSST}
	
	% Include both collaboration papers and external citations:
	\bibliography{main,lsstdesc}
	
\end{document}

% ======================================================================