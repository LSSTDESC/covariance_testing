\RequirePackage{docswitch}
% \flag is set by the user, through the makefile:
%    make note
%    make apj
% etc.
\setjournal{\flag}

\documentclass[\docopts]{\docclass}

%%%% Scott's macros
\newcommand{\sfig}[2]{
\includegraphics[width=#2]{#1}
        }
\newcommand{\Sfig}[2]{
    \begin{figure}[thbp]
    \sfig{../Figures/#1.pdf}{\columnwidth}
    \caption{{\small #2}}
    \label{fig:#1}
    \end{figure}
}
\newcommand{\Swide}[2]{
\begin{figure*}[thbp]
 \sfig{../Figures/#1.pdf}{.8\textwidth}
  \caption{{\small #2}}
   \label{fig:#1}
   \end{figure*}
}
\newcommand{\Sswide}[2]{
\begin{figure*}[thbp]
 \sfig{../Figures/#1.pdf}{.7\textwidth}
  \caption{{\small #2}}
   \label{fig:#1}
   \end{figure*}
}
\newcommand{\Svwide}[2]{
\begin{figure*}[thbp]
 \sfig{../Figures/#1.pdf}{\textwidth}
  \caption{{\small #2}}
   \label{fig:#1}
   \end{figure*}
}

\newcommand{\Spng}[2]{
    \begin{figure}[thbp]
    \sfig{../Figures/#1.png}{0.95\columnwidth}
    \caption{{\small #2}}
    \label{fig:#1}
    \end{figure}
}
\newcommand{\Rf}[1]{\ref{fig:#1}}
\newcommand{\rf}[1]{\ref{fig:#1}}
\newcommand{\ec}[1]{Eq.~(\ref{eq:#1})}
\newcommand{\ecalt}[1]{Eq.~\ref{eq:#1}}
\newcommand{\Ec}[1]{(\ref{eq:#1})}
\newcommand{\eeec}[3]{Eqs.~(\ref{eq:#1}, \ref{eq:#2}, \ref{eq:#3})}
\newcommand{\eql}[1]{\label{eq:#1}}
\newcommand\be{\begin{equation}}
\newcommand\ee{\end{equation}}
\def\bea{\begin{eqnarray}}
\def\eea{\end{eqnarray}}
% \def\bea{\begin{eqnarray}}
% \def\eea{\end{eqnarray}}
\def\svs{\nonumber\\}

% You could also define the document class directly
%\documentclass[]{emulateapj}

% Custom commands from LSST DESC, see texmf/styles/lsstdesc_macros.sty
\usepackage{lsstdesc_macros}

\usepackage{graphicx}
\graphicspath{{./}{./figures/}}
\bibliographystyle{apj}

% Add your own macros here:



% ======================================================================

\begin{document}

\title{Covariance Testing}

\maketitlepre

\begin{abstract}

There are a number of codes that compute covariance matrices analytically; the plan is to use these to build TJPCov. In this project, we start along the path of comparing these different codes, building up a suite of tools that can be used to compare covariance matrices. We expect these tools to be useful not only for converging on a single accurate code for computing covariance matrices but also more generally for understanding which parts of the covariance matrix carry the most information (and therefore need the most attention to get right) and which are not relevant (so for example matrices that are not positive definite may still be usable if the negative eigenmodes are not relevant).
\end{abstract}

% Keywords are ignored in the LSST DESC Note style:
\dockeys{}

\maketitlepost

% ----------------------------------------------------------------------
% 

\section{Introduction}
\label{sec:intro}


% ----------------------------------------------------------------------

\section{Methods}
\label{sec:methods}

There are several ways to tests covariance matrices. We will illustrate each of these on cosmic shear statistics $\xi_\pm(\theta)$, focusing for the most part of the Year 1 results of the Dark Energy Survey~\cite{Abbott:2017wau}. One of the codes will be {\tt Cosmolike}~\cite{Krause:2016jvl}; another will be the one used to analyze the KiDS-450 survey~\cite{Kohlinger:2017sxk}.


\subsection{One-to-one Comparison}

This is a simple matter of comparing elements of a covariance matrix, usually starting with diagonal elements. \figref{xipmscatter} shows an example.

\begin{figure}
\includegraphics[width=0.9\columnwidth]{xipmscatter.png}
\caption{A simple scatter plot of elements of covariance matrices produced by two separate halo model codes. \label{fig:xipmscatter}}
\end{figure}


\subsection{Eigenvalues and eigenvectors}

This is slightly more sophisticated: diagonalize the covariance matrix and examine the eigenvalues and also the associated eigenvectors. \figref{coveigen} shows an example of the eigenvalues from two different covariance codes.

\begin{figure}
\includegraphics[width=0.9\columnwidth]{coveigen.png}
\caption{A simple scatter plot of elements of covariance matrices produced by two separate halo model codes. \label{fig:coveigen}}
\end{figure}

\figref{evector} shows an example of one of the eigenvectors, the one associated with the smallest eigenvalue. This low-eigenvalue mode picks up the differences between the correlation function at different angular scales (each vertical line delineates between two-point functions of shears in different tomographic bin pairs).

\begin{figure}
\includegraphics[width=0.9\columnwidth]{evector.png}
\caption{A simple scatter plot of elements of covariance matrices produced by two separate halo model codes. \label{fig:evector}}
\end{figure}

\subsection{Parameter Estimation}

Ultimately, what matters is how well the likelihood does at extracting parameter constraints. Since most analyses assume a Gaussian likelihood, this boils down to how well the contours in parameter space agree when computing the $\chi^2$ using two different covariance matrices.

\subsection{Shrinkage}

There have been several methods proposed in the literature to compress the data vectors, extracting as much information as possible. Here we consider two: first compression at the map level~\cite{Alonso:2017hhj}, where linear combination of the tomographic maps are used. If there are 4 tomographic bins, an uncompressed analysis would require ten separate 2-point functions (or 20 for cosmic shear), whereas a compression scheme leads to just a few uncorrelated maps. If there were 3 such maps, then only three 2-point functions would need to be used for the likelihood analysis.

We characterize a given element in the data vector by its angular and tomographic indices: $a_i = a_{lm,\alpha}$ where $l,m$ denote the indices corresponding to given spherical harmonics and $\alpha$ is a given tomographic bin, or equivalently $a_i = a_\alpha(\vec\theta)$ in real space. The compression occurs in terms of tomographic bins, so that 
\be
b_\mu(\vec\theta) = \sum_{\alpha} F_{\mu\alpha} a_\alpha(\theta)
\ee
where the $F$'s are chosen\footnote{Note that this definition of $F$ differs from that in ~\cite{Alonso:2017hhj} in that it includes the inverse of the noise matrix.} so that the correlation functions of the $b$'s are diagonal in tomographic space: 
\bea
w_{\mu\nu}(\theta) &=& \langle b_\mu(\vec\theta_1) b_\nu(\vec\theta_2)\rangle\vert_{\vert\vec\theta_1-\vec\theta_2\vert\in\theta}
\svs
&=& \delta_{\mu,\nu} w_{\mu}(\theta).
\eea
Usually with $N_t$ tomographic bins, one must consider $N_t(N_t+1)/2$ correlation functions, but -- due to the transformation that diagonalizes the elements -- there are only $N_t$ correlation functions to consider (for a given $\theta$). Even better, these can be ordered by the information they contain, so fewer than $N_t$can be used. In the example used in ~\cite{Alonso:2017hhj}, 16 tomographic bins were assumed, so that the standard treatment would require 136 correlation functions, but only 3 were needed in order to extract accurate constraints. 

This method suggests a way of extracting the most important pieces of the full covariance matrix $C$. We simply compute the covariance matrix of the $w_{\mu}$ in terms of $C$ and keep only the most important terms. That is,
\bea
C^b_{\mu\nu} &\equiv& \langle (w_{\mu} -\bar w_\mu)\,(w_{\nu} -\bar w_\nu)\rangle
\svs
&=& \sum_{\alpha\alpha'\beta\beta'} F_{\mu\alpha}F_{\mu\alpha'}\, F_{\nu\beta}F_{\nu\beta'}\, C_{\alpha\alpha'\beta\beta'} .
\eea
Here the angular indices have been suppressed (there are two of them, one for each $w_\mu$), but the full tomographic complexity has been retained. The covariance matrix on the right includes a total of $N_t^2 \times N_t^2$ terms (some of which are equal because of symmetry) corresponding to all possible pairs of two-point functions. But $C^b$ on the right contains only $N_t^2$ elements, and again these are ordered, so we can use only a subset of them. In the simplest case, where only one linear combination is needed so $\mu=\nu=1$, a single number (for each pair of angular bins) captures all the relevant information from the full covariance matrix.

The second compression takes place at the 2-point level~\cite{Zablocki:2015zcm}, with the compressed data vector containing linear combinations of the many 2-point functions. In principle, this might work with only $N_p$ 2-point functions where $N_p$ is the number of parameters varied, and each mode, or linear combination, contains all the information necessary about the parameter of interest. 

For each parameter $p_\alpha$ that is varied, one captures a single linear mode
\be
y_\alpha = U_{\alpha,i} D_i
\ee
where $D_i$ are the data points and the coefficients are defined as
\be
U_{\alpha,i} \equiv \frac{\partial T_j}{\partial p_\alpha} \, C^{-1}{}_{ji}
\ee
where $T_j$ is the theoretical prediction for the data point $D_j$.
The now much smaller data set $\{y_\alpha\}$, which contains as few as $N_p$ data points carries with it its own covariance matrix, with which the $\chi^2$ can be computed for each point in parameter space. Propagating through shows that this covariance matrix is related to the original $C_{ij}$ via
\be
C_{\alpha\beta} = U_{\alpha i} C_{ij} U^t_{j\beta}.
\ee
To take an example from DES-Y1, $C_{ij}$ is a $400\times 400$ matrix, while the number of parameters needed to specify the model is only 16, so $C_{\alpha\beta}$ is a $16\times 16$ matrix. We have apparently captured from the initial set of $400\times 401/2=80,200$ independent elements of the covariance matrix a small subset (only 136 in this case) of linear combinations of these 80k elements that really matter. If two covariance matrices give the same set of $C_{\alpha\beta}$, we do not care whether any of the other eighty thousand elements differ from one another.


% ----------------------------------------------------------------------

\section{Results}
\label{sec:results}



% ----------------------------------------------------------------------

\section{Discussion}
\label{sec:discussion}



% ----------------------------------------------------------------------

\section{Conclusion}
\label{sec:conclusion}



% ----------------------------------------------------------------------

\subsection*{Acknowledgments}

%%% Here is where you should add your specific acknowledgments, remembering that some standard thanks will be added via the \code{desc-tex/ack/*.tex} and \code{contributions.tex} files.

%This paper has undergone internal review in the LSST Dark Energy Science Collaboration. % REQUIRED if true

	%T.F. and T.Z. contributed equally on writing the main paper as well as implementing the covariance comparison and compression. N.C. contributed to the compression code. All authors participated in the discussion and gave valuable suggestions.
	
	The contributions are listed below. T.F. contributed to the manuscript, led the analysis for the eigenvalues, SNR and MOPED, as well as the comparison of the compressed covariance matrices. T.Z. contributed to the manuscript, participated and contributed substantially to all analysis, and led the analysis for the tomographic compression. N.C. contributed to the compression code. S.D. proposed the project, the analyses, led the discussions and also contributed to writing and editing the manuscript. % Standard papers only: author contribution statements. For examples, see http://blogs.nature.com/nautilus/2007/11/post_12.html

% This work used TBD kindly provided by Not-A-DESC Member and benefitted from comments by Another Non-DESC person.

% Standard papers only: A.B.C. acknowledges support from grant 1234 from ...

\input{desc-tex/ack/standard} % also available: key standard_short

% This work used some telescope which is operated/funded by some agency or consortium or foundation ...

% We acknowledge the use of An-External-Tool-like-NED-or-ADS.

%{\it Facilities:} \facility{LSST}

% Include both collaboration papers and external citations:
\bibliography{main,lsstdesc}

\end{document}

% ======================================================================
